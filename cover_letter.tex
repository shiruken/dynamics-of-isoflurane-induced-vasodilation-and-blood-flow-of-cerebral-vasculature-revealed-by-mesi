\documentclass[12pt,stdletter,orderfromtodate,dateleft,sigleft]{newlfm}
\usepackage[hidelinks]{hyperref}

\topmarginsize{0in}
\topmarginskip{0in}
\headermarginsize{0in}
\headermarginskip{0in}

\leftmarginsize{1in}
\rightmarginsize{1in}

\newlfmP{addrfromskipbefore=0pt}
\newlfmP{addrfromskipafter=0pt}

\newlfmP{sigskipbefore=10pt}
\newlfmP{sigsize=40pt}

\newlfmP{Headlinewd=0pt,Footlinewd=0pt}

\namefrom{Colin Sullender, Ph.D.}
\addrfrom{%
    Colin Sullender\\
    The University of Texas at Austin\\
    Department of Biomedical Engineering\\
    107 W. Dean Keeton Street, Stop C0800\\
    Austin, TX, 78712, USA
}

\dateset{\today}

\greetto{Dear Dr. Boas,}

\closeline{Sincerely,}

\begin{document}
\begin{newlfm}

We wish to submit a new manuscript entitled ''Visualizing anesthesia-induced vasodilation of cerebral vasculature using multi-exposure speckle imaging'' for consideration by \emph{Neurophotonics}. We confirm that this work is original and has not been published elsewhere nor is it currently under consideration in another journal.

In this paper, we present cerebral blood flow measurements in awake mice during the induction of general anesthesia with isoflurane. We utilize multi-exposure speckle imaging (MESI) to highlight the large anatomical changes caused by the vasodilatory inhalant. We also characterize the impact of anesthesia on cortical hemodynamics across multiple subjects and imaging sessions.

This work is significant because it provides a unique insight into the acute effects of general anesthesia. Existing studies have largely focused on discrete measurements before and after the induction of anesthesia whereas we present wide-field imagery during the induction itself. These results provide further evidence that neuroscience experiments would benefit from transitioning to un-anesthetized awake animal models.

Please address all correspondence to Dr. Andrew Dunn at \href{mailto:adunn@utexas.edu}{\underline{adunn@utexas.edu}}.

Thank you for your consideration.

\end{newlfm}
\end{document}